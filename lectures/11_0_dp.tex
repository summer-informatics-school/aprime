\chapter{Динамика}

\section{Семинар по ДП} 

Оценка за задачу равна (количеству человек, не сдавших эту задачу, плюс 1), умноженному на понижающий коэффициент подпункта, если он указан.

\begin{enumerate}
\item Наибольшая общая возрастающая подпоследовательность 
двух последовательностей
    \begin{enumerate}
    \item $O(n^4)$
    \item $O(n^3)$
    \item $O(n^2)$
    \end{enumerate}
\item
Последовательность длины $n$ называется {\it выпуклой}, 
если 
$\forall i \colon 1 < i < n \colon a_i < \frac{(a_{i - 1} + a_{i + 1})}{2}$. 

Найдите наибольшую выпуклую подпоследовательность данной 
последовательности за $O(n ^ 3)$.

\item

{\it Разорванным числом} называется последовательность из 
ровно $n$ цифр от 1 до 9 такая, что любые две соседние цифры 
отличаются не менее чем на 3. Упорядочим все разорванные числа 
по возрастанию. Найдите по заданному разорванному числу его номер
в этом списке. $O(9\cdot n)$
\item
Найдите минимальное $k$-ичное число  
    \begin{enumerate}   
    \item с заданной суммой $S$ последовательных попарных произведений цифр за $O(S \cdot 10^2)$.
    \item с заданной суммой $S$ попарных произведений цифр как можно быстрее
    \end{enumerate}
\item
Дана произвольная скобочная последовательность из круглых \texttt{( )},
 квадратных \texttt{[ ]}, фигурных \texttt{\{ \}} и угловых \texttt{< >} скобок. 
 Требуется удалить из неё минимальное количество скобок,
  чтобы она стала правильной (и скобки соответствовали скобкам
   того же типа). $O(n^3)$
\item
Дано дерево. Выведите за $O(n)$:
    \begin{enumerate}
    \item для каждого из рёбер "--- количество простых путей, 
    проходящих через него
    \item для каждого из рёбер "--- сумму длин простых путей, 
    проходящих через него
    \item для каждого из рёбер "--- сумму квадратов длин простых путей,
    проходящих через него
    \end{enumerate}
\item
В языке <<Kava>> есть $n$ программных конструкций, 
длина каждой из которых не превосходит $m$. 
Валидная программа на языке <<Kava>> "--- это произвольная 
последовательность конструкций, каждую конструкцию можно 
использовать неограниченное число раз. 
Сколько различных программ длины $L$ можно написать 
на языке <<Kava>>? $O(min(n, m)\cdot L)$
\item
Проверить два корневых дерева на изоморфизм
\item
Задача о рюкзаке. Есть $n$ слитков золота, 
каждый имеет вес $w_i$ и рюкзак, который выдерживает вес $W$.
\begin{enumerate}
    \item Требуется положить в рюкзак как можно больше золота:
    \begin{enumerate}
        \item ($\times 0.2$) Найти только вес за $O(W \cdot n)$, память $O(W \cdot n)$
        \item ($\times 0.2$) Найти только вес за $O(W \cdot n)$, память $O(W)$
        \item ($\times 0.2$) Найти сам способ за $O(W \cdot n)$, память $O(W \cdot n)$
        \item($\times 0.5$) Найти сам способ за $O(W \cdot n)$, память $O(W)$
    \end{enumerate}
    \item ($\times 0.2$) А теперь всё то же самое, 
    но каждый слиток можно брать бесконечное число раз.
    \item Теперь каждого типа слитка $k_i$ штук. 
    Сделайте всё то же самое за время:
    \begin{enumerate}
        \item ($\times 0.2$) $O(W \times \sum k_i)$
        \item ($\times 0.5$) $O(W \times \sum log k_i)$
    \end{enumerate}
    \item ($\times 0.2$) Теперь каждый из слитков весит 
    не больше $m$, и запас снова бесконечен. 
    Решите ту же задачу за время $O(n \cdot m \cdot log (m))$.
\end{enumerate}
\item
Найти какую-либо строчку минимальной длины, 
подходящую под два заданных шаблона (с вопросиками и звёздочками), 
или определить, что такой нет. $O(L_1 \cdot L_2)$ времени и памяти.
\item 
Дана последовательность из $N$ чисел.  Пусть $\sum$ "--- сумма чисел на подотрезке.
    \begin{enumerate}
    \item ($\times 0.2$) Подотрезок с максимальной суммой за $O(N)$.
    \item ($\times 0.2$) Подотрезок длины от $L$ до $R$ с максимальной суммой за $O(N)$. 
    \item ($\times 0.5$) Три непересекающихся подотрезка такие, что $\sum_1 + \sum_2 + \sum_3$ максимально за $O(N)$.
    \item ($\times 0.5$) Пункт а на окружности.
    \item ($\times 0.5$) Пункт b на окружности.
    \item ($\times 1$) Пункт c на окружности.
    \end{enumerate}
\item 
$O(N^3)$. Дан прямоугольник, состоящий из $N \times N$ целых чисел. 
    \begin{enumerate}
    \item Выбрать подпрямоугольник \t{max} суммы
    \item Выбрать два непересекающихся подпрямоугольника \t{max} суммы. 
    \item Выбрать три непересекающихся подпрямоугольника \t{max} суммы. 
    \item Выберите подпрямоугольник с \t{max} суммой в четырёх углах
    \item Выберите подпрямоугольник с \t{max} суммой по периметру
    \end{enumerate}
\item 
Найдите число различных подпоследовательностей в последовательности 
длины $n$, составленной из чисел от 1 до $n$ за $O(n)$.
\end{enumerate}
          
