\chapter{ Паросочетания }
\section{Теорема об удлиняющей цепочке}
\begin{Def}
Граф называется \textit{двудольным}, если существует его правильная раскраска в два цвета, то
есть вершины можно разбить на две доли так, чтобы ребра шли лишь между вершинами разных долей.
\end{Def}

\begin{Def}
\textit{Паросочетанием} называется любое множество ребер, не имеющих общих концов.
\end{Def}

\begin{Def}
\textit{Максимальным паросочетанием} называется любое максимальное по размеру (количнству ребер) паросочетание.
\end{Def}

\begin{Def}
Пусть фиксировано некоторое паросочетание. Вершина называется \textit{свободной}, если нет ни одного ребра паросочетания,
смежного с данной вершиой. В противном случае вершина называется \textit{занятой}.
\end{Def}

\begin{Def}
Пусть фиксировано некоторое паросочетание. \textit{Удлиняющей или чередующейся цепочкой} называется 
любой путь из свободной вершины в свободную,
ребра в котором чередуются: ребро не из паросочетания, ребро из паросочетания и т.д.
\end{Def}

\begin{theorem}{(об удлиняющей цепочке)}
Паросочетание максимально тогда и только тогда, когда в графе нет удлиняющей цепочки.
\end{theorem}

\begin{proof}
$\Ra$ Если в графе есть удлиняющая цепочка, то выбросив ребра этой цепочки из паросочетания и добавив
ребра этой цепочки, которые до этого ему не принадлежали, мы увеличим размер паросочетания на 1.

$\La$ Теперь докажем, что если паросочетание не максимально, то существует удлиняющая цепочка. Действительно,
пусть $M$~--- текущее паросочетание, а $M'$~--- максимальное. Тогда рассмотрим \textit{симметрическую разность}
$M \triangle M'$ двух паросочетаний как множеств ребер, а именно рассмотрим ребра, лежащие ровно в одном из паросочетаний.

Рассмотрим граф, содержащий только рассматриваемые ребра из симметрической разности. В этом графе степень каждой вершины
не превосходит 2, так как каждой вершине может быть инцидентно максимум одно ребро каждого из паросочетаний.
Все такие графы имеют очень простой вид: это набор цепей и циклов. При этом, так как $|M'|>|M|$, то найдется компонента связности,
в которой ребер из $M'$ больше. Эта компонента является цепью. Концы этой цепи являются свободными вершинами в $M$ (иначе
мы бы добавили в симметрическую разность еще одно ребро). Таким образом, данная цепь и есть искомая удлиняющая цепочка.
\end{proof}

\section{Алгоритм Куна}
Алгоритм: найдем удлиняющую цепь, увеличим паросочетание... 

Будем искать цепочки только из левой доли, потому что цепочка заканчивается в разных долях. 

\begin{cppcode}
int curTime;

bool dfs(v) {
    if(used[v] == curTime) return false; //очевидно, что бессмысленно
    //ходить в одну вершину два раза.
    used[v] = curTime; 
    for u: (v, u) in E {
        if (pair[u] == -1 || dfs(pair[u])) {
            pair[u] = v;
            return true;
        }
    }
    return false;
}

int main() {
    for (int i = 1; i <= n; ++i) {
        if (dfs(i)) {
            cnt++;
            ++curTime;
        }
    }
}
\end{cppcode}
   
Такой алгоритм будет работать.

Если вершина насытилась, то она уже не перестанет быть насыщенной в дальнейшем. 
Пусть из i мы не нашли паросочетание, значит, мы уже не найдём паросочетание с этой вершиной, то есть, 
когда мы находим удлиняющую цепочку, то цепочка состоит из вершин, не достижимых из i.

Оптимизация от Сережи.

Можно обнулять used только когда мы нашли новую цепь.  

Теперь, почему не нужно проверять, что i ненасыщенное. Потому что до этого мы должны были прийти из меньший вершины, 
но тогда эта вершина должна была быть насыщенна до этого. 


\textbf{Задача:} Пусть вершины левой доли взвешены, нужно найти максимальное по весу паросочетание. 

\begin{proof}
Нужно отсортировать вершины в порядке убывания. 

Пусть мы запустили Куна и для первых i паросочетание оптимально. 
\begin{enumerate}
    \item от i + 1 вершины Кун не нашел паросочетание. Пусть есть более хорошее паросочетание для i + 1,
    тогда i + 1  вершина участвует. Единственный вариант, когда мы могли что-то улучшить, это с цепочкой нечётной длины,
    иначе мы выкинули какую-то более жирную вершину.  
    \item Если добавили i + 1 вершину, то из старого паросочетания мы ничего не выкинули, значит все хорошо. Иначе бы паросочетание для первых i 
    было бы не оптимально. 
\end{enumerate}
\end{proof}

\section{Вершинное покрытие и независимое множество}
\begin{Def}
Вершинное покрытие "--- множество вершин, которое покрывает каждое ребро хотя бы одной вершиной.
\end{Def}
\begin{Def}
Независимое множество "--- это множество несмежных вершин.
\end{Def}

Нам интересно максимальное независимое множеcтво и минимальное вершинное покрытие.

\begin{theorem}{}
Дополнение любого вершинного покрытия "--- независимое множество.
\end{theorem}
\begin{proof}
Пусть дополнение зависимое, тогда есть вершины соединённые ребром, тогда это ребро 
никто не покрывает
\end{proof}

\begin{theorem}{}
Пусть есть какое-то максимальное паросочетание. 

Размер вершинного покрытия $\ge$ максимального паросочетания. 
\end{theorem}
\begin{proof}

У всех паросочетаний мы должны взять хотя бы одну вершину, мы одной вершиной не можем покрыть два ребра из паросочетания. 

\end{proof}

\begin{theorem}{Теорема Кенинга}

Минимальное вершинное покрытие = максимальному паросочетанию.     

\end{theorem}

\begin{proof}
  
M "--- максимальное паросочетание

Сориентируем ребра: Ребра из паросочетания справа налево, не из паросочетания слева направо.

Запустим dfs из ненасыщенных вершин левой доли. 

То что обошли $A^+$ и $B^+$ и то что не посетили $A^{-}$ и $B^-$. Вершины левой доли: $A$, правой "---  $B$. 

Заметим, что между $A^{+}$ и $B^{-}$ вообще нет ребер. Иначе бы у нас была удлиняющая цепочка.
$A^{+}$ и $B^{-}$ независимое множество, а дополнение вершинное покрытие.

Посчитаем размер $$|B^+ \cup A^{-}| = |МП|$$ все вершины из $A^-$ насыщенные, иначе бы мы с них начали. Любая вершинка из $B^{+}$
насыщенная, иначе бы у нас была удлиняющая цепочка. $$|B^+ \cup A^-|\le |MП|$$ Значит это минимальное вершинное покрытие.

\end{proof}

\section{Цепи и антицепи}


G "--- ациклический транзитивно замкнутый ориентированный граф.

\begin{Def}
Антицепь "--- Множество вершин из G, такое, что любые 2 вершины не соединены.
\end{Def}
                                                                 
Дан G, хотим найти максимальную антицепь.

\begin{theorem}{Теорема Дилворта}
Размер максимальной антицепи равен минимальному количеству вершино непересекающихся цепей, покрывающих все вершины графа.
\end{theorem}

\begin{proof}                                              
\begin{enumerate}
\item  Давайте расклеим граф. У каждой вершины будет копия и проведем ребра из $u$ в $v'$, граф получился двудольный. 
\item  Найдем максимальное паросочетание в этом графе.
\item  Найдем минимаьное вершинное покрытие. 
\item  (u, u') не лежат одновременно в вершинном покрытие
\item  $A = V \smallsetminus (u| u \in min ВП \cup u' \in min ВП)$, A "--- антицепь.

От противного:

Пусть есть вершины u и v соединённые ребром в антицепи, тогда это ребро никто не покрывает, противоречие.
\item A "--- максимальная антицепь.

$$|A| = N - K$$

Выделим ребра из паросочетания в начальном графе. Начальный граф можно покрыть $N - K$ цепями. 

$$|МП| = N - K$$ 

Пойдем по ребрам паросочетания и получится цепь. Цепей будет ровно столько, сколько ненасыщенных вершин в правой доле. 

\item Никакая цепочка не может проходить через две вершинки антицепи. Значит, что бы покрыть все вершины антицепи нужно цепей не меньше, чем 
длина антицепи.

\item Наша антицепь максимальна. Пусть есть антицепь $$|A'| > |A| = N - K$$ тогда две вершинки попали в одну цепь, тогда это не атницепь. 
\end{enumerate}
\end{proof}