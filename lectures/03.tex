\chapter{Корневая эвристика}
\section{sqrt-декомпозиция}
Рассмотрим несколько задач:
\begin{enumerate}
\item  Дан массив чисел, требуется выполнять запросы:
    \begin{enumerate}
    \item добавить x на [L, R)
    \item найти сумму на [L, R)
    \end{enumerate}
    
    \begin{proof}
    
    Разобьём массив на блоки размера k.

    Предподсчитаем сумму в блоках. 
    
    Теперь, что бы ответить на запросы второго типа, нужно посчитать сумму на всех 
    блоках, а потом отдельно посчитать сумму на кусочках. Время работы $O(k + \frac{N}{k})$

    Что бы добавить на отрезке, заведем вспомогательный массив add, прибавка на блоках. Время работы $O(k + \frac{N}{k})$

    k выберем равным $\sqrt{N}$.

    В задачах обычно уже даны ограничения, и k можно брать константой, с константой можно экспериментировать. 

    \begin{cppcode}
    for(i = L; i < R; ) {
        if (i % k == 0 && i + k <= R) {
            res += sum[i/k];
            i += k;
        } else {
            res += a[i] + add[i/k];
            ++i;
        }
    }
    \end{cppcode}
    \end{proof}
\item 
    Дан массив чисел, требуется выполнять запросы:
    \begin{enumerate}
    \item прибавить x на [L, R)
    \item есть ли элемент, равный x на [L, R)
    \end{enumerate} 

    \begin{proof}
    \begin{enumerate}
    \item Разобьём на блоки длины k, в каждом блоке будем хранить мультисет

    \begin{cppcode}
    multiset<int> S;
    s.erase(x);//удаляет все x
    s.erase(s.find(x)); //удалить один x
    \end{cppcode}

    При ответе на запрос делаем find в нужных блоках и в остальных пробегаемся втупую.

    Когда делаем добавку, нужно искать $x - add_i$ 

   \item Другое решение "--- хранить отсортированные блоки массива и там искать бинпоиском. На двух блоках, где что-то 
    испортили втупую отсортируем куски. $O(\frac{N}{k} + k \log (k))$

    Можно отсортировать куски за линию. Разделить кусок на две части "--- префикс и суффикс, к одной части прибавили, 
    а дальше merge. Теперь прибавление работает за $O(\frac{N}{k} + k)$

    Ответ на запрос работает за $O(k + \frac{N}{k} \log k)$

    $k = \sqrt{N \log N}$

    $$\sqrt{N \log N} + \frac{N \log {\sqrt {N\log (N)}}}{ \sqrt{N \log N}} =
    \sqrt{N} \frac{\frac12(\log N + \log \log N)}{\sqrt{\log N}} = \sqrt{N \log N}$$
   \end{enumerate}
   \end{proof}

\item Массив, заполнен неотрицательными числами
    \begin{enumerate}
    \item вычитаем x на [L, R)
    \item $\forall$ элемента найти, на какой операции он стал не больше 0.
    \end{enumerate} 

    \begin{proof}
    \begin{enumerate}
    
    \item Храним отсортированную версию блоков и бинпоиском ищем первый 0 и храним, где у нас начинались неотрицательные числа.

    $O(\sqrt{N \log N})$

    Можно бинпоиск заменить на указатели.

    $O(\sqrt{N})$
    
    \item Другое решение. У нас есть Q запросов, разобьём запросы на блоки по k.

    После k запросов $A \to A'$, можем найти элементы, которые были положительные и стали отрицательные. Для этих элементов
    локально в блоке ищем момент, когда этот элемент стал отрицательный
    
    Для построения массива $A'$ у нас будет вспомогательный массив pref, когда отрезок появляется делаем -x, когда кончается +x.

    swap(A, A')

    Время работы $O(\frac{Q}{k}(N + k) + N \cdot k)$ 
    $$k = \sqrt{Q}, O(N \sqrt{Q} + Q)$$
    \end{enumerate}
    \end{proof}
\item Неориентированный граф (V, E). Каждая вершина имеет цвет от $1 \cdots k$.

Запросы:
    \begin{enumerate}
    \item Перекрасить вершину v в цвет c.
    \item Найти количества различных цветов среди соседей.
    \end{enumerate} 
    \begin{proof}

    Можно в тупую хранить цвет, и пробегать по всем соседям и считать количество разных вершин. Но это слишком наивное решение.

   Заметим, что если окрестность у соседей маленькая, то можно решить задачу в тупую.
   Если количество соседей у вершины больше $k$, то вершину назовем тяжелой.

   Всего количество тяжелых вершин $\le \frac{2m}{k}$.

   Заранее сохраним ответ для тяжелых вершин. Храним массив: количество вершин данного цвета и 
   переменную для вершины: количество разных цветов.

   Ответ на запрос за $O(k)$, запрос на изменение за $O(\frac{2m}{k})$.
   \end{proof}
\end{enumerate}

\chapter{Hash-таблицы}
    \begin{enumerate}
    \item Закрытая адресация. 
    
    $s \to h(s) \sim$ 64 бита.
    
     $M = 2^{20}$ создадим массив списков размера M.

     add(S):
     \begin{enumerate}
     \item \cpp"h(s) % M = x"
     \item Берем список x и добавляем в него строку s.
     \item Если хотим найти строку s, то ищем ее в соответствующем списке.
     \end{enumerate}

     $M = 2^{20}$ для более быстрого взятия по модулю.

     M должно быть раза в два-три больше N. Можно перестраивать хештаблицу по факту, если какой-то список стал очень большой, 
     то стоит сделать хештаблицу в два раза больше.
    \item Открытая адресация.
    
    \begin{enumerate}
    \item Есть просто массив объектов. Посчитали hash по модулю M, пусть он равен x. Пытаемся положить объект в ячейку x, если свободно, то кладем. Если в ячейке занято, займем ближайшую справа ячейку. 
    \item Для поиска идем вправо, пока все ячейки заняты и пока не встретили нашу строку идем в следующую 
    ячейку.
    \item Просто так удалить нельзя.
    \item Можно пометить ячейку, как плохую. И если плохих стало очень много, перестроим таблицу.
    \item Для следующий ячейки смотрим важно ли, что ячейка была занята, переносим элемент и рекурсивно удаляем. 
    \end{enumerate}

    \end{enumerate}

\chapter{Битовое сжатие}

\textbf{Задача о рюкзаке:}
   
    W "--- максимальная вместимость. 
   
    $w_1, \cdots, w_k$ "--- веса предметов.

    Хотим найти максимальный вес меньше W, который можно 
    набрать с помощью этих предметов.

\begin{proof}    
    
    \cpp"dp[i][j] = dp[i - 1][j] | dp[i - 1][j - w[i]]"

    Время работы $O(kW)$

    \cpp"dp[i] = dp[i - 1] | (dp[i - 1] >> w[i])"

    Можно руками превратить в массив int и так сделать or.

    \cpp"bitset <N> b;" 

    \cpp"N" "--- размер битсета 

    \cpp"b[i]" "--- обратиться к элементу

    \cpp"b << k" сдвинуть битсет на k.

    Рюкзак:
    
    \begin{cppcode}
        bitset<S + 1> f;
        f[0] = 1;
        forn(i, n) 
            f |= f << a[i]; // O(S/word)
    \end{cppcode}
\end{proof}